% Options for packages loaded elsewhere
\PassOptionsToPackage{unicode}{hyperref}
\PassOptionsToPackage{hyphens}{url}
%
\documentclass[
]{book}
\usepackage{amsmath,amssymb}
\usepackage{lmodern}
\usepackage{ifxetex,ifluatex}
\ifnum 0\ifxetex 1\fi\ifluatex 1\fi=0 % if pdftex
  \usepackage[T1]{fontenc}
  \usepackage[utf8]{inputenc}
  \usepackage{textcomp} % provide euro and other symbols
\else % if luatex or xetex
  \usepackage{unicode-math}
  \defaultfontfeatures{Scale=MatchLowercase}
  \defaultfontfeatures[\rmfamily]{Ligatures=TeX,Scale=1}
\fi
% Use upquote if available, for straight quotes in verbatim environments
\IfFileExists{upquote.sty}{\usepackage{upquote}}{}
\IfFileExists{microtype.sty}{% use microtype if available
  \usepackage[]{microtype}
  \UseMicrotypeSet[protrusion]{basicmath} % disable protrusion for tt fonts
}{}
\makeatletter
\@ifundefined{KOMAClassName}{% if non-KOMA class
  \IfFileExists{parskip.sty}{%
    \usepackage{parskip}
  }{% else
    \setlength{\parindent}{0pt}
    \setlength{\parskip}{6pt plus 2pt minus 1pt}}
}{% if KOMA class
  \KOMAoptions{parskip=half}}
\makeatother
\usepackage{xcolor}
\IfFileExists{xurl.sty}{\usepackage{xurl}}{} % add URL line breaks if available
\IfFileExists{bookmark.sty}{\usepackage{bookmark}}{\usepackage{hyperref}}
\hypersetup{
  pdftitle={Klinikleitfaden},
  pdfauthor={Klinik und Poliklinik für Anästhesiologie und Intensivtherapie},
  hidelinks,
  pdfcreator={LaTeX via pandoc}}
\urlstyle{same} % disable monospaced font for URLs
\usepackage{longtable,booktabs,array}
\usepackage{calc} % for calculating minipage widths
% Correct order of tables after \paragraph or \subparagraph
\usepackage{etoolbox}
\makeatletter
\patchcmd\longtable{\par}{\if@noskipsec\mbox{}\fi\par}{}{}
\makeatother
% Allow footnotes in longtable head/foot
\IfFileExists{footnotehyper.sty}{\usepackage{footnotehyper}}{\usepackage{footnote}}
\makesavenoteenv{longtable}
\usepackage{graphicx}
\makeatletter
\def\maxwidth{\ifdim\Gin@nat@width>\linewidth\linewidth\else\Gin@nat@width\fi}
\def\maxheight{\ifdim\Gin@nat@height>\textheight\textheight\else\Gin@nat@height\fi}
\makeatother
% Scale images if necessary, so that they will not overflow the page
% margins by default, and it is still possible to overwrite the defaults
% using explicit options in \includegraphics[width, height, ...]{}
\setkeys{Gin}{width=\maxwidth,height=\maxheight,keepaspectratio}
% Set default figure placement to htbp
\makeatletter
\def\fps@figure{htbp}
\makeatother
\setlength{\emergencystretch}{3em} % prevent overfull lines
\providecommand{\tightlist}{%
  \setlength{\itemsep}{0pt}\setlength{\parskip}{0pt}}
\setcounter{secnumdepth}{5}
\usepackage{booktabs}
\ifluatex
  \usepackage{selnolig}  % disable illegal ligatures
\fi
\usepackage[]{natbib}
\bibliographystyle{apalike}

\title{Klinikleitfaden}
\usepackage{etoolbox}
\makeatletter
\providecommand{\subtitle}[1]{% add subtitle to \maketitle
  \apptocmd{\@title}{\par {\large #1 \par}}{}{}
}
\makeatother
\subtitle{Leitfaden perioperative Patientenbetreuung}
\author{Klinik und Poliklinik für Anästhesiologie und Intensivtherapie}
\date{2021-05-27}

\begin{document}
\maketitle

{
\setcounter{tocdepth}{1}
\tableofcontents
}
\hypertarget{vorwort}{%
\chapter*{Vorwort}\label{vorwort}}
\addcontentsline{toc}{chapter}{Vorwort}

wer nie sein Brot mit Tränen ass
Der Leitfaden dient der Orientierung von Kollegen, die neu in eine Abteilung wechseln, dort als Springer eingesetzt werden oder im Akutschmerzdienst tätig sind.

Er informiert über abteilungsspezifische Abläufe/ Standards und beantwortet häufig gestellte Fragen.

Er ist nicht bindend und entbindet gleichzeitig den/die Anästhesist/in nicht von einer individuellen Risikoevaluation und entsprechender Adaptation des perioperativen anästhesiologischen Managements und der postoperativen Schmerztherapie.

Eine Beachtung von evtl. Kontraindikationen und Höchstdosierungen (individuelle Dosierungsanpassung für einzelne empfohlene Medikamente) wird erwartet.

Ebenso sind unabhängig von dem Leitfaden die Dienstanweisungen zu berücksichtigen.

Die Bearbeiter der einzelnen Kapitel sind verantwortlich für deren Inhalt.

zuletzt erzeugt am 27 Mai, 2021

\hypertarget{hinweise-zur-benutzung}{%
\chapter*{Hinweise zur Benutzung}\label{hinweise-zur-benutzung}}
\addcontentsline{toc}{chapter}{Hinweise zur Benutzung}

\begin{itemize}
\tightlist
\item
  Die Listen der Prozeduren/ Operationen innerhalb jedes Bereichs sind alphabetisch sortiert. Die Informationen zu den einzelnen Bereichen sind bewusst teilweise sehr knapp gehalten. \textbf{Weitere Informationen stehen in ANE-Wiki.}
\end{itemize}

\textbf{Beachte:}

\begin{itemize}
\item
  Es wurden Hyperlinks zu den Wiki-Seiten, zum „Springen" innerhalb des Dokumentes, ins Intranet und ins Internet eingepflegt.
\item
  Die Reanimationsalgorithmen des ERC befinden sich in den hinteren Umschlagseiten.
\item
  Der Schockraumalgorithmus befindet sich in der vorderen Umschlagseite.
\item
  \textbf{Besonders möchten wir auf eGena hinweisen (siehe S. 5) -- Eine Hilfe auch für Erfahrene!}
\item
  Gedruckte Anleitungen unterliegen, sobald sie Druckpresse verlassen haben, einem rapiden Alterungsprozess. Aktuelle Neuerungen - dies betrifft insbesondere die SOPs - werden nach der Freigabe durch die Klinikleitung im Intranet veröffentlicht. Dort besteht auch die Möglichkeit, eine PDF-Version dieses Leitfadens herunter zu laden.
\end{itemize}

M. Hübler

05.01.2021

\hypertarget{anwendungsmuxf6glichkeiten-im-notfall}{%
\section*{Anwendungsmöglichkeiten im Notfall}\label{anwendungsmuxf6glichkeiten-im-notfall}}
\addcontentsline{toc}{section}{Anwendungsmöglichkeiten im Notfall}

Überprüfen von Handlungsschritten auf Vollständigkeit
Vorlesen der Handlungsschritte und Ausführung durch das Team

\hypertarget{anwendungsmuxf6glichkeiten-im-alltag}{%
\section*{Anwendungsmöglichkeiten im Alltag}\label{anwendungsmuxf6glichkeiten-im-alltag}}
\addcontentsline{toc}{section}{Anwendungsmöglichkeiten im Alltag}

\begin{itemize}
\tightlist
\item
  Selbststudium
\item
  Grundlage für Unterricht oder Mentoring
\item
  Vorbesprechung (‚Briefing`) möglicher Komplikationen bei einem konkreten Patienten
\item
  Nachbesprechung (‚Debriefing`) eines Zwischenfalls
\end{itemize}

\hypertarget{ane-wiki}{%
\section*{ANE-WIKI}\label{ane-wiki}}
\addcontentsline{toc}{section}{ANE-WIKI}

Auf den \href{https://carusshare.uniklinikum-dresden.de/kiz/ane/anewiki/Wiki/Homepage.aspx}{ANE-Wiki} Seiten finden sich viele nützliche Informationen für den Klinikalltag, aber auch zu den einzelnen Bereichen. Die Seiten leben davon, dass möglichst viele Informationen eingetragen und entsprechend aktualisiert werden. Veränderungen eintragen kann jeder. Wer dazu Fragen hat, bitte an Prof.~Hübler wenden.
Auf ANE-Wiki kann auch von außerhalb des Intranets zugegriffen werden kann. Am einfachsten funktioniert dies über den Internetauftritt unserer Klinik über die Seite \url{http://www.uniklinikum-dresden.de}. Dort über Kliniken Anästhesie anwählen. Links befindet sich der Link zu Carusshare „von extern``. Die Anmeldung entspricht der persönlichen Anmeldung an UKD-Rechnern 

\hypertarget{part-i-anuxe4sthesie-bereiche}{%
\part{I Anästhesie Bereiche}\label{part-i-anuxe4sthesie-bereiche}}

\hypertarget{aop-h-51}{%
\chapter{AOP (H 51)}\label{aop-h-51}}

T. Müller

\hypertarget{wichtige-telefonnummern}{%
\section{Wichtige Telefonnummern}\label{wichtige-telefonnummern}}

\begin{itemize}
\tightlist
\item
  Bereichsleiter: 19910
\item
  OP-Saal 1 (Andok 55): 6915
\item
  OP-Saal 2 (Andok 56): 6914
\item
  OP-Saal 3 (Andok 57): 6913
\item
  OP-Saal 4 (Andok 58): 6911
\item
  ANE-Pflege: 19361/ 19362
\item
  Aufenthaltsraum: 6975
\item
  Tagesklinik (Sr.~Gabi): 18168
\item
  Reinigung: 19341
\item
  OP-Büro: 6966\\
\item
  OP-Pflege (Sr.~A. Blum): 16966
\end{itemize}

\hypertarget{ausstattung}{%
\section{Ausstattung}\label{ausstattung}}

\begin{itemize}
\tightlist
\item
  Narkosegerät Leon
\item
  C-Mac
\item
  Einmalbrochoskop 5 mm
\item
  Hotline
\item
  BZ-Gerät in Tagesklinik
\item
  Ultraschall (GE-Laptop, Mindray Laptop)
\item
  Relaxometer
\item
  Nervenstimulator
\item
  Baxter Pumpen
\item
  Rea-Rucksack
\item
  LifePak 20e
\end{itemize}

\hypertarget{patientenstruxf6me}{%
\section{Patientenströme}\label{patientenstruxf6me}}

\hypertarget{normalstation}{%
\subsection{Normalstation}\label{normalstation}}

\begin{itemize}
\tightlist
\item
  erster Patient kommt automatisch
\item
  Folgepatienten möglichst im Austausch mit AWR
\item
  AWR als Holdingarea nutzen.
\end{itemize}

\hypertarget{mk3-s2-diabetische-fuxfcuxdfe}{%
\subsection{MK3-S2 (diabetische Füße)}\label{mk3-s2-diabetische-fuxfcuxdfe}}

\begin{itemize}
\tightlist
\item
  Die Patienten werden zuvor von der Station in die Tagesklinik (Sr.~Gabi) gebracht.
\item
  post-OP ohne AWR wieder in die TK.
\item
  isolationspflichtige Patienten absprechen.
\end{itemize}

\hypertarget{isolationspflichtige-patienten}{%
\subsection{isolationspflichtige Patienten}\label{isolationspflichtige-patienten}}

\begin{itemize}
\tightlist
\item
  direkt in den Saal, aus dem Saal direkt auf Station
\item
  Parallele Einleitung nur bei freiem Saal nach Absprache
\end{itemize}

\hypertarget{ambulante-patienten}{%
\subsection{ambulante Patienten}\label{ambulante-patienten}}

\begin{itemize}
\tightlist
\item
  werden aus TK abgerufen
\end{itemize}

\hypertarget{kinder}{%
\section{Kinder}\label{kinder}}

\begin{itemize}
\tightlist
\item
  Voraussetzungen: \textgreater10 Jahre und \textgreater30 kg
\item
  keine Risikokinder im AOP (z.B. Dysmorphie, schwieriger Atemweg, ITS-Pflichtigkeit, unkooperative Patienten, etc.) Risikopatienten
\item
  keine Risikopatienten im AOP, insbesondere:

  \begin{itemize}
  \tightlist
  \item
    hohe Transfusionswahrscheinlichkeit (kein BGA-Gerät)
  \item
    OP mit wahrscheinlicher postoperativer ITS-/IMC-Pflichtigkeit
  \item
    bekannter schwieriger Atemweg nach Rücksprache
  \end{itemize}
\end{itemize}

\hypertarget{patienten-von-its}{%
\section{Patienten von ITS}\label{patienten-von-its}}

\begin{itemize}
\tightlist
\item
  Voraussetzung:

  \begin{itemize}
  \tightlist
  \item
    stabiler Kreislauf
  \item
    keine schwerwiegende respiratorische Störung
  \end{itemize}
\end{itemize}

\hypertarget{auflegen}{%
\section{Auflegen}\label{auflegen}}

\begin{itemize}
\tightlist
\item
  Schultertisch nur für Schulterchirurgie
\item
  bei Fuß-OP: Füße bis ans Ende des OP-Tisches
\item
  2 Armschienen
\item
  1 angeschraubter Infusionsständer
\end{itemize}

\hypertarget{einleitung}{%
\section{Einleitung}\label{einleitung}}

\begin{itemize}
\tightlist
\item
  OP-Tisch für RA nicht anbremsen
\item
  Patienten mit Name im Monitor aufnehmen: Scanner
\item
  niedrige Tropfengeschwindigkeit der Infusion
\item
  bei Standardeingriffen keine 3-Wege-Hähne oder Verlängerungen
\item
  Monitoring-Kabel sollen nicht den Boden berühren
\end{itemize}

\hypertarget{anuxe4sthesie-bei-ambulanten-operationen}{%
\section{Anästhesie bei ambulanten Operationen}\label{anuxe4sthesie-bei-ambulanten-operationen}}

\begin{itemize}
\tightlist
\item
  beachte allg. \href{https://www.reanitrain.de/downloads/leitlinien/operative\%20Eingriffe/ambulante\%20OPs\%20-\%20zahnaerztliche\%20Eingriffe/Leitlinie\%20fuer\%20ambulantes\%20Operieren\%20bzw.Tageschirurgie.pdf}{Empfehlung DGAI}
\item
  sparsam infundieren
\item
  Spinalanästhesie

  \begin{itemize}
  \tightlist
  \item
    Chlorprocain, Prilocain 2\% oder Bupivacain 0,5\% ohne Fentanyl
  \item
    nach einer Spinalanästhesie: Ultraschall der Harnblase im AWR (Dokumentation nicht vergessen!)
  \end{itemize}
\item
  Regionalanästhesie obere Extremität

  \begin{itemize}
  \tightlist
  \item
    Bei zu erwartenden postoperativen Schmerzen Ropivacain sonst Prilocain
  \end{itemize}
\item
  Regionalanästhesie untere Extremität

  \begin{itemize}
  \tightlist
  \item
    Ropivacain nur nach Rücksprache
  \item
    Wundrandinfiltration durch den Operateur mit Ropivacain falls möglich
  \end{itemize}
\item
  vor Entlassung: post-OP Visite durch Anästhesie (Tagesklinik)
\end{itemize}

\hypertarget{aug}{%
\chapter{AUG}\label{aug}}

M. Trausch

\hypertarget{wichtige-telefonnummern-1}{%
\section{Wichtige Telefonnummern}\label{wichtige-telefonnummern-1}}

\begin{itemize}
\tightlist
\item
  Bereichsleiter: 18048 (Trausch)
\item
  ANE-Einleitung: 2114
\item
  ANE-Pflege: 11802
\item
  AUG-Aufwachraum: 2687
\item
  Station AUG-S1: 13451 (3451)
\item
  Station AUG-S4: 12352 (2352)
\item
  Vorbereiter (AUG-Arzt): 12114
\item
  OP-Pflege Saal 4 (ITNs) 11819
\end{itemize}

\hypertarget{allgemeines-augenklinik}{%
\section{Allgemeines-Augenklinik}\label{allgemeines-augenklinik}}

\hypertarget{pruxe4medikation}{%
\subsection{Prämedikation}\label{pruxe4medikation}}

\begin{itemize}
\tightlist
\item
  Augenklinik bestimmt i.d.R. KEINE Routinelaborwerte (bei Erfordernis „Bitte Labor!{}`` vermerken).
\item
  Antihypertensiva (Dauermedikation) BEIBEHALTEN
\end{itemize}

\hypertarget{anuxe4sthesiefuxfchrung}{%
\subsection{Anästhesieführung}\label{anuxe4sthesiefuxfchrung}}

\begin{itemize}
\tightlist
\item
  postoperative Schmerztherapie mit peripheren Analgetika ausreichend
\item
  AWR: Montags kein Anspruch auf ANE-Pflegekraft für AWR
\item
  Besonderheiten Stationäre Kinder aus Kinderklinik
\item
  Ambulante Patienten Mi + Do im Haus
\end{itemize}

\hypertarget{eingriffe}{%
\section{Eingriffe}\label{eingriffe}}

\hypertarget{dmek-descement-mebrane-endothelial-keratoplasty}{%
\subsection{DMEK (Descement Mebrane Endothelial Keratoplasty)}\label{dmek-descement-mebrane-endothelial-keratoplasty}}

\begin{itemize}
\tightlist
\item
  Prämedikation: keine Besonderheiten
\item
  Anästhesieführung LM/ ITN
\item
  Zugänge/Monitoring Standard
\end{itemize}

\hypertarget{enukleation}{%
\subsection{Enukleation}\label{enukleation}}

\begin{itemize}
\tightlist
\item
  Beschreibung (kurz): Entfernung des Bulbus oculi\\
\item
  Beschreibung (lang):

  \begin{itemize}
  \tightlist
  \item
    Bei einer Enukleation wird der Bulbus oculi entfernt. Indikationen sind Tumoren (z.B. Retinoblastom, Melanom), Phthisis bulbi und gelegentlich direkte Traumafolgen. Bei der OP werden das hinter dem Auge befindliche Bindegewebe und die Augenmuskeln belassen. Der Augapfel wird meist durch eine Plombe ersetzt und an dieser die Augenmuskeln teilweise fixiert. So kann sich die später eingesetzte Augenprothese teilweise zum gesunden Auge mitbewegen.
  \item
    Bei einer Exenteration des Auges werden zusätzlich auch das Bindegewebe und die Augenmuskeln entfernt.
  \end{itemize}
\item
  Prämedikation: keine Besonderheiten
\item
  Anästhesieführung: ITN
\item
  Zugänge/Monitoring: Standard
\end{itemize}

\hypertarget{glaukom-op}{%
\subsection{Glaukom-OP}\label{glaukom-op}}

\begin{itemize}
\tightlist
\item
  Beschreibung (lang):

  \begin{itemize}
  \tightlist
  \item
    Zyklophotokoagulation (ZPK)

    \begin{itemize}
    \tightlist
    \item
      Hier wird durch eine gezielte Laserbehandlung ein Teil des Ziliarkörpers vernarbt. Im Ziliarkörper wird das Kammerwasser produziert; durch eine teilweise Vernarbung erreicht man, dass weniger Kammerwasser produziert wird, und der Augendruck damit sinkt.
    \item
      Der Eingriff dauert nur wenige Minuten und kann in örtlicher Betäubung durchgeführt werden. Ggf. wird eine Allgemeinanästhesie mit Larynxmaske durchgeführt.
    \end{itemize}
  \item
    Trabekelektomie (TE)

    \begin{itemize}
    \tightlist
    \item
      Hier wird ein künstlicher Abflussweg für das Kammerwasser zu einer vorpräparierten Bindehauttasche geschaffen. Damit der neue Abfluss durch ablaufende Reparaturprozesse (Verwachsung, Narben) nicht wieder verschlossen wird, kommt Mitomycin C (MMC) zum Einsatz.
    \item
      Operationszeit ca. 40 min.
    \end{itemize}
  \item
    Ahmed-Implantat

    \begin{itemize}
    \tightlist
    \item
      Hier handelt es sich um ein Drainagesystem, das über eine Mikro-Schlauchverbindung Kammerwasser von der Vorderkammer des Auges zu einem unter die Bindehaut eingesetzten Implantat leitet, von wo das Kammerwasser über die Bindehaut resorbiert wird.
    \item
      Die Operationszeit beträgt ca. 45 min.
    \end{itemize}
  \item
    Preserflo-Implantat

    \begin{itemize}
    \tightlist
    \item
      Implantation eines Mikro-Schlauches, der Kammerwasser aus der Vorderkammer in eine vorpräparierte Bindehauttasche ableitet.
    \end{itemize}
  \end{itemize}
\item
  Beschreibung (kurz):

  \begin{itemize}
  \tightlist
  \item
    ZPK, Zyklokryotherapie, TE, Ahmed-Implantat, InnFocus-(Preserflo-)Implantat
  \end{itemize}
\item
  Prämedikation: Zyklophotokoagulation (ZPK) bitte auf orale PM verzichten
\item
  Anästhesieführung: LAMA, ITN
\item
  Zugänge/Monitoring: Standard
\end{itemize}

\hypertarget{katarakt-op}{%
\subsection{Katarakt-OP}\label{katarakt-op}}

\begin{itemize}
\tightlist
\item
  Beschreibung (lang):

  \begin{itemize}
  \tightlist
  \item
    Bei der Kararakt-OP (bei uns umgangssprachlich mit ``Phako'' oder ``Phako/HKL'' bezeichnet) wird die getrübte Linse durch ein künstliches Linsenimplantat (siehe Abbildung) ersetzt. Der Eingriff erfolgt meist in Retrobulbäranästhesie. Falls eine Anästhesie erforderlich ist, wird meist eine Larynxmaske selten eine ITN durchgeführt.
  \item
    OP-Dauer bis ca. 10 min (bei schwierigen Verhältnissen bis 40 min).
  \end{itemize}
\item
  Prämedikation: keine Besonderheiten
\item
  Anästhesieführung: LAMA, ITN
\item
  Zugänge/Monitoring: Standard
\end{itemize}

\hypertarget{keratoplastik-perforierende-ane-wiki}{%
\subsection{Keratoplastik, perforierende ANE-Wiki}\label{keratoplastik-perforierende-ane-wiki}}

\begin{itemize}
\tightlist
\item
  Prämedikation: keine Besonderheiten
\item
  Anästhesieführung: ITN
\item
  Zugänge/Monitoring: Standard
\item
  Besonderheit: 150 mg Prednisolon, 500 mg Acetazolamid
\end{itemize}

\hypertarget{narkoseuntersuchung-ane-wiki}{%
\subsection{Narkoseuntersuchung ANE-Wiki}\label{narkoseuntersuchung-ane-wiki}}

\begin{itemize}
\tightlist
\item
  Prämedikation: keine Besonderheiten
\item
  Anästhesieführung: LAMA, ITN
\item
  Zugänge/Monitoring: Standard
\end{itemize}

\hypertarget{ppv-kryolimbusparallele-plombe-ane-wiki}{%
\subsection{PPV, kryo/limbusparallele Plombe ANE-Wiki}\label{ppv-kryolimbusparallele-plombe-ane-wiki}}

\begin{itemize}
\tightlist
\item
  Prämedikation: keine Besonderheiten
\item
  Anästhesieführung: ITN
\item
  Zugänge/Monitoring: Standard
\end{itemize}

\hypertarget{schiel-op-ane-wiki}{%
\subsection{Schiel-OP ANE-Wiki}\label{schiel-op-ane-wiki}}

\begin{itemize}
\tightlist
\item
  Prämedikation: keine Besonderheiten
\item
  Anästhesieführung: LAMA, ITN
\item
  Zugänge/Monitoring: Standard
\item
  Besonderheiten: Glycopyrronium, Granisetron
\end{itemize}

\hypertarget{truxe4nenwegs-op-ane-wiki}{%
\subsection{Tränenwegs-OP ANE-Wiki}\label{truxe4nenwegs-op-ane-wiki}}

\begin{itemize}
\tightlist
\item
  Beschreibung:

  \begin{itemize}
  \tightlist
  \item
    Kinder: Spülung u/o. Schlingenintubation;
  \item
    Erw.: oft als Dakryozystorhinostomie (DCR)
  \end{itemize}
\item
  Prämedikation: keine Besonderheiten
\item
  Anästhesieführung: LM, DCR immer ITN
\item
  Zugänge/Monitoring: Standard
\item
  Besonderheiten: bei DCR Rachentamponade
\end{itemize}

\hypertarget{applications}{%
\chapter{Applications}\label{applications}}

Some \emph{significant} applications are demonstrated in this chapter.

\hypertarget{example-one}{%
\section{Example one}\label{example-one}}

\hypertarget{example-two}{%
\section{Example two}\label{example-two}}

\hypertarget{final-words}{%
\chapter{Final Words}\label{final-words}}

We have finished a nice book.

  \bibliography{book.bib,packages.bib}

\end{document}
