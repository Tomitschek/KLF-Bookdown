% Options for packages loaded elsewhere
\PassOptionsToPackage{unicode}{hyperref}
\PassOptionsToPackage{hyphens}{url}
%
\documentclass[
]{book}
\usepackage{amsmath,amssymb}
\usepackage{lmodern}
\usepackage{ifxetex,ifluatex}
\ifnum 0\ifxetex 1\fi\ifluatex 1\fi=0 % if pdftex
  \usepackage[T1]{fontenc}
  \usepackage[utf8]{inputenc}
  \usepackage{textcomp} % provide euro and other symbols
\else % if luatex or xetex
  \usepackage{unicode-math}
  \defaultfontfeatures{Scale=MatchLowercase}
  \defaultfontfeatures[\rmfamily]{Ligatures=TeX,Scale=1}
\fi
% Use upquote if available, for straight quotes in verbatim environments
\IfFileExists{upquote.sty}{\usepackage{upquote}}{}
\IfFileExists{microtype.sty}{% use microtype if available
  \usepackage[]{microtype}
  \UseMicrotypeSet[protrusion]{basicmath} % disable protrusion for tt fonts
}{}
\makeatletter
\@ifundefined{KOMAClassName}{% if non-KOMA class
  \IfFileExists{parskip.sty}{%
    \usepackage{parskip}
  }{% else
    \setlength{\parindent}{0pt}
    \setlength{\parskip}{6pt plus 2pt minus 1pt}}
}{% if KOMA class
  \KOMAoptions{parskip=half}}
\makeatother
\usepackage{xcolor}
\IfFileExists{xurl.sty}{\usepackage{xurl}}{} % add URL line breaks if available
\IfFileExists{bookmark.sty}{\usepackage{bookmark}}{\usepackage{hyperref}}
\hypersetup{
  pdftitle={Klinikleitfaden},
  pdfauthor={Klinik und Poliklinik für Anästhesiologie und Intensivtherapie},
  hidelinks,
  pdfcreator={LaTeX via pandoc}}
\urlstyle{same} % disable monospaced font for URLs
\usepackage{longtable,booktabs,array}
\usepackage{calc} % for calculating minipage widths
% Correct order of tables after \paragraph or \subparagraph
\usepackage{etoolbox}
\makeatletter
\patchcmd\longtable{\par}{\if@noskipsec\mbox{}\fi\par}{}{}
\makeatother
% Allow footnotes in longtable head/foot
\IfFileExists{footnotehyper.sty}{\usepackage{footnotehyper}}{\usepackage{footnote}}
\makesavenoteenv{longtable}
\usepackage{graphicx}
\makeatletter
\def\maxwidth{\ifdim\Gin@nat@width>\linewidth\linewidth\else\Gin@nat@width\fi}
\def\maxheight{\ifdim\Gin@nat@height>\textheight\textheight\else\Gin@nat@height\fi}
\makeatother
% Scale images if necessary, so that they will not overflow the page
% margins by default, and it is still possible to overwrite the defaults
% using explicit options in \includegraphics[width, height, ...]{}
\setkeys{Gin}{width=\maxwidth,height=\maxheight,keepaspectratio}
% Set default figure placement to htbp
\makeatletter
\def\fps@figure{htbp}
\makeatother
\setlength{\emergencystretch}{3em} % prevent overfull lines
\providecommand{\tightlist}{%
  \setlength{\itemsep}{0pt}\setlength{\parskip}{0pt}}
\setcounter{secnumdepth}{5}
\usepackage{booktabs}
\ifluatex
  \usepackage{selnolig}  % disable illegal ligatures
\fi
\usepackage[]{natbib}
\bibliographystyle{apalike}

\title{Klinikleitfaden}
\usepackage{etoolbox}
\makeatletter
\providecommand{\subtitle}[1]{% add subtitle to \maketitle
  \apptocmd{\@title}{\par {\large #1 \par}}{}{}
}
\makeatother
\subtitle{Leitfaden perioperative Patientenbetreuung}
\author{Klinik und Poliklinik für Anästhesiologie und Intensivtherapie}
\date{2021-04-22}

\begin{document}
\maketitle

{
\setcounter{tocdepth}{1}
\tableofcontents
}
\hypertarget{vorwort}{%
\chapter*{Vorwort}\label{vorwort}}
\addcontentsline{toc}{chapter}{Vorwort}

Der Leitfaden dient der Orientierung von Kollegen, die neu in eine Abteilung wechseln, dort als Springer eingesetzt werden oder im Akutschmerzdienst tätig sind.

Er informiert über abteilungsspezifische Abläufe/ Standards und beantwortet häufig gestellte Fragen.

Er ist nicht bindend und entbindet gleichzeitig den/die Anästhesist/in nicht von einer individuellen Risikoevaluation und entsprechender Adaptation des perioperativen anästhesiologischen Managements und der postoperativen Schmerztherapie.

Eine Beachtung von evtl. Kontraindikationen und Höchstdosierungen (individuelle Dosierungsanpassung für einzelne empfohlene Medikamente) wird erwartet.

Ebenso sind unabhängig von dem Leitfaden die Dienstanweisungen zu berücksichtigen.

Die Bearbeiter der einzelnen Kapitel sind verantwortlich für deren Inhalt.

\hypertarget{hinweise-zur-benutzung}{%
\chapter*{Hinweise zur Benutzung}\label{hinweise-zur-benutzung}}
\addcontentsline{toc}{chapter}{Hinweise zur Benutzung}

\begin{itemize}
\tightlist
\item
  Die Listen der Prozeduren/ Operationen innerhalb jedes Bereichs sind alphabetisch sortiert. Die Informationen zu den einzelnen Bereichen sind bewusst teilweise sehr knapp gehalten. \textbf{Weitere Informationen stehen in ANE-Wiki.}
\end{itemize}

\textbf{Beachte:}

\begin{itemize}
\item
  Es wurden Hyperlinks zu den Wiki-Seiten, zum „Springen" innerhalb des Dokumentes, ins Intranet und ins Internet eingepflegt.
\item
  Die Reanimationsalgorithmen des ERC befinden sich in den hinteren Umschlagseiten.
\item
  Der Schockraumalgorithmus befindet sich in der vorderen Umschlagseite.
\item
  \textbf{Besonders möchten wir auf eGena hinweisen (siehe S. 5) -- Eine Hilfe auch für Erfahrene!}
\item
  Gedruckte Anleitungen unterliegen, sobald sie Druckpresse verlassen haben, einem rapiden Alterungsprozess. Aktuelle Neuerungen - dies betrifft insbesondere die SOPs - werden nach der Freigabe durch die Klinikleitung im Intranet veröffentlicht. Dort besteht auch die Möglichkeit, eine PDF-Version dieses Leitfadens herunter zu laden.
\end{itemize}

M. Hübler

05.01.2021

\hypertarget{part-part-i}{%
\part{Part I}\label{part-part-i}}

\hypertarget{asa-klassifikation}{%
\chapter{ASA-Klassifikation}\label{asa-klassifikation}}

\hypertarget{aufkluxe4rung}{%
\chapter{Aufklärung}\label{aufkluxe4rung}}

\hypertarget{cervicalblock}{%
\section{Cervicalblock}\label{cervicalblock}}

\begin{itemize}
\tightlist
\item
  Infektion
\item
  Gefäßverletzung (Hämatom)
\item
  Nervenverletzung
\item
  peridurale Ausbreitung
\item
  Recurrenslähmung
\end{itemize}

\hypertarget{erector-spinae-katheter-esk}{%
\section{Erector spinae Katheter (ESK)}\label{erector-spinae-katheter-esk}}

\begin{itemize}
\tightlist
\item
  Infektion
\item
  Gefäßverletzung (Hämatom)
\item
  Nervenverletzung
\item
  Pneumothorax
\end{itemize}

\hypertarget{femoralisblock-fnb--katheter}{%
\section{Femoralisblock (FNB)/ -katheter}\label{femoralisblock-fnb--katheter}}

\begin{itemize}
\tightlist
\item
  Infektion
\item
  Gefäßverletzung (Hämatom)
\item
  Femoralisschaden
\item
  Lokalanästhetikaintoxikation
\end{itemize}

\hypertarget{intubationsnarkose-doppellumen}{%
\section{Intubationsnarkose (Doppellumen)}\label{intubationsnarkose-doppellumen}}

\begin{itemize}
\tightlist
\item
  siehe Intubationsnarkose (oral)
\item
  Trachealverletzung (Ruptur)
\item
  Bronchusverletzung
\end{itemize}

\hypertarget{intubationsnarkose-oral}{%
\section{Intubationsnarkose (oral)}\label{intubationsnarkose-oral}}

\begin{itemize}
\tightlist
\item
  Zahnschäden/ Schäden an der Kauleiste bei Zahnprothesenträgern
\item
  Heiserkeit/ Halsschmerzen
\item
  Kehlkopf-/ Stimmbandverletzung
\item
  Übelkeit/ Erbrechen/ Aspiration
\item
  Awareness
\item
  allergische Reaktion auf Medikamente
\item
  Kreislaufstörungen
\item
  Hypoxie
\end{itemize}

\hypertarget{intubationsnarkose-nasal}{%
\section{Intubationsnarkose (nasal)}\label{intubationsnarkose-nasal}}

\begin{itemize}
\tightlist
\item
  siehe Intubationsnarkose (oral)
\item
  Epistaxis (Nasenbluten)
\end{itemize}

\hypertarget{kaudalanuxe4sthesie--katheter}{%
\section{Kaudalanästhesie/ -katheter}\label{kaudalanuxe4sthesie--katheter}}

\begin{itemize}
\tightlist
\item
  siehe Periduralanästhesie
\item
  Lokalanästhetikaintoxikation
\end{itemize}

\hypertarget{larynxmaske---tubus}{%
\section{Larynxmaske/ - tubus}\label{larynxmaske---tubus}}

\begin{itemize}
\tightlist
\item
  siehe Intubationsnarkose (oral)
\item
  Sensibilitätsstörungen an der Zunge
\end{itemize}

\hypertarget{lidocain-perfusor}{%
\section{Lidocain-Perfusor}\label{lidocain-perfusor}}

\begin{itemize}
\tightlist
\item
  Off-label use
\item
  Herzrhythmusstörungen, Kammerflimmern
\item
  Lokalanästhetika-Intoxikation
\end{itemize}

\hypertarget{maske}{%
\section{Maske}\label{maske}}

\begin{itemize}
\tightlist
\item
  siehe Intubationsnarkose (wg. mögl. Eskalation)
\end{itemize}

\hypertarget{pcia}{%
\section{PCIA}\label{pcia}}

\begin{itemize}
\tightlist
\item
  Übelkeit
\item
  Obstipation
\item
  Harnverhalt
\item
  Müdigkeit
\end{itemize}

\hypertarget{peniswurzelblock}{%
\section{Peniswurzelblock}\label{peniswurzelblock}}

\begin{itemize}
\tightlist
\item
  Infektion
\item
  Hämatom
\item
  penile Sensibilitätsstörungen (Nervenschaden)
\end{itemize}

\hypertarget{periduralanuxe4sthesie}{%
\section{Periduralanästhesie}\label{periduralanuxe4sthesie}}

\begin{itemize}
\tightlist
\item
  siehe Spinalanästhesie
\item
  Duraperforation
\end{itemize}

\hypertarget{plexusanuxe4sthesie-axilluxe4r}{%
\section{Plexusanästhesie (axillär)}\label{plexusanuxe4sthesie-axilluxe4r}}

\begin{itemize}
\tightlist
\item
  Nervenschaden
\item
  Infektion
\item
  Gefäßverletzung/ Hämatom
\item
  Lokalanästhetikaintoxikation
\item
  Versagen des Verfahrens
\end{itemize}

\hypertarget{plexusanuxe4sthesie-infraclaviculuxe4r}{%
\section{Plexusanästhesie (infraclaviculär)}\label{plexusanuxe4sthesie-infraclaviculuxe4r}}

\begin{itemize}
\tightlist
\item
  siehe Plexusanästhesie (axillär)
\item
  Pneumothorax
\end{itemize}

\hypertarget{plexusanuxe4sthesie-supraclaviculuxe4r}{%
\section{Plexusanästhesie (supraclaviculär)}\label{plexusanuxe4sthesie-supraclaviculuxe4r}}

\begin{itemize}
\tightlist
\item
  siehe Plexusanästhesie (axillär)
\item
  Phrenicuslähmung/-parese
\item
  Recurrenslähmung (Heiserkeit)
\item
  Horner-Syndrom
\item
  je nach Verfahren peridurale Ausbreitung
\end{itemize}

\hypertarget{psoaskompartmentblock--katheter}{%
\section{Psoaskompartmentblock/ -katheter}\label{psoaskompartmentblock--katheter}}

\begin{itemize}
\tightlist
\item
  Infektion
\item
  Gefäßpunktion/ Hämatom
\item
  Nervenschaden (Hüftadduktorenschwäche, Knie-extensorenschwäche)
\item
  peridurale Ausbreitung
\item
  Lokalanästhetikaintoxikation
\end{itemize}

\hypertarget{pulmonaliskatheter-swan-ganz-katheter}{%
\section{Pulmonaliskatheter (Swan-Ganz-Katheter)}\label{pulmonaliskatheter-swan-ganz-katheter}}

\begin{itemize}
\tightlist
\item
  siehe \protect\hyperlink{zvk}{ZVK}
\item
  Kammerflimmern
\item
  Lungenembolie
\item
  Lungeninfarkt
\end{itemize}

\hypertarget{transversus-abdominis-plane-block-tap}{%
\section{Transversus abdominis plane block (TAP)}\label{transversus-abdominis-plane-block-tap}}

\begin{itemize}
\tightlist
\item
  Infektion
\item
  Hämatom
\item
  Lokalanästhetikaintoxikation
\end{itemize}

\hypertarget{spinalanuxe4sthesie}{%
\section{Spinalanästhesie}\label{spinalanuxe4sthesie}}

\begin{itemize}
\tightlist
\item
  Infektion
\item
  Hämatom
\item
  Nervenschaden inkl. Querschnittslähmung, Parästhesein
\item
  postspinale Kopfschmerzen
\item
  transitorisches neurologisches Defizit
\item
  Harnverhalt
\item
  Infektion
\item
  Fehlpunktion (arteriell, Hämatom)
\item
  Pneumothorax
\item
  Nervenverletzung
\item
  Herzrhythmusstörungen
\end{itemize}

\hypertarget{zvk}{%
\section{ZVK}\label{zvk}}

\begin{itemize}
\tightlist
\item
  Infektion
\item
  Fehlpunktion (arteriell, Hämatom)
\item
  Pneumothorax
\item
  Nervenverletzung
\item
  Herzrhythmusstörungen
\end{itemize}

\hypertarget{applications}{%
\chapter{Applications}\label{applications}}

Some \emph{significant} applications are demonstrated in this chapter.

\hypertarget{example-one}{%
\section{Example one}\label{example-one}}

\hypertarget{example-two}{%
\section{Example two}\label{example-two}}

\hypertarget{final-words}{%
\chapter{Final Words}\label{final-words}}

We have finished a nice book.

  \bibliography{book.bib,packages.bib}

\end{document}
